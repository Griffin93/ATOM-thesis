%---------------------------------------------------------------------------%
%->> 封面信息及生成
%---------------------------------------------------------------------------%
%-
%-> 中文封面信息
%-
\confidential{}% 密级:只有涉密论文才填写
\schoollogo{scale=0.5}{nwu_logo}% 校徽
\title{低分辨率环境下的微表情识别}% 论文中文题目
\author{李桂锋}% 论文作者
\advisor{\hspace{+3.0em}彭进业~教授~西北大学}% 指导教师:姓名 专业技术职务 工作单位
%\advisorsec{}% 指导老师附加信息 或 第二指导老师信息
\degree{硕士}% 学位:学士、硕士、博士
\degreetype{工学}% 学位类别:理学、工学、工程、医学等
\major{电子与通信工程}% 二级学科专业名称
\institute{信息科学与技术学院}% 院系名称
\chinesedate{2019~年~6~月}% 毕业日期:夏季为6月、冬季为12月
%-
%-> 英文封面信息
%-
\englishtitle{Micro-expression Recognition\\Under Low-resolution Case}% 论文英文题目
\englishauthor{Li Guifeng}% 论文作者
\englishadvisor{Supervisor: Peng Jinye Professor}% 指导教师
\englishdegree{Master}% 学位:Bachelor, Master, Doctor。封面格式将根据英文学位名称自动切换,请确保拼写准确无误
\englishdegreetype{Engineering}% 学位类别:Philosophy, Natural Science, Engineering, Economics, Agriculture 等
\englishthesistype{thesis}% 论文类型: thesis, dissertation
\englishmajor{Electronics and Communication Engineering}% 二级学科专业名称
\englishinstitute{ }% 院系名称
\englishdate{June 2019}% 毕业日期:夏季为June、冬季为December
%-
%-> 生成封面
%-
\maketitle% 生成中文封面
\makeenglishtitle% 生成英文封面
%-
%-> 作者声明
%-
\makedeclaration% 生成声明页
%-
%-> 中文摘要
%-
\chapter{摘\quad 要}\chaptermark{摘\quad 要}% 摘要标题
\setcounter{page}{1}% 开始页码
\pagenumbering{Roman}% 页码符号

人脸表情在我们的社交互动中发挥着重要作用,因为它传达了丰富的信息。我们可以从一张人脸图像中阅读很多内容,但是如果没有特殊设备,我们也无法感知到这些信息。本文采用计算机视觉方法分析肉眼难以察觉的微妙的面部信息:微表情。

微表情是快速、不自主的面部表情,属于一种基本的非言语行为,它揭示了人们不打算表达的情感。人们很难感知微表情,因为它们太快和微弱,因此自动微表情分析是很有价值的工作,具有重大的应用前景。例如它在国家安全、计算机辅助诊断等领域有着广泛的应用,这促使我们对自动微表情识别进行研究。然而,从监视视频捕获的图像容易遭受低质量问题,这导致微表情识别算法在实际应用中产生困难。由于捕获图像的质量较低,现有算法无法达到预期的效果。为了解决这个问题,我们用人脸Hallucination方法对低分辨率环境下的微表情识别问题进行了全面的研究。本文综述了微表情研究的进展,并分四部分进行描述。(1)我们介绍了目前存在的微表情数据集,着重介绍了第一个自发的微表情数据集——SMIC。数据的缺乏阻碍了微表情的分析研究。由于很难收集自发的微表情,所以引入用于诱导和注释SMIC的协议以帮助将来对微表情数据的收集。 (2)介绍了基于传统机器学习算法的低分辨率环境下的微表情识别框架,其在两个成熟的数据集——SMIC和CASME上进行了实验,实验结果表明,所提出的框架在低分辨率情况下获得了有良好的识别结果,这结束了现有算法无法识别低分辨率微表情的处境。 (3)采用数据增强和数据集融合的方法对现有数据集进行扩充,引入了时下最先进的针对图像序列的卷积神经网络——P3D ResNet网络从深度特征层面识别低分辨率的微表情,该方法与基于传统机器学习的算法相比,准确度得到进一步提升。 (4)提出了一种低分辨率环境下微表情识别系统——LMERS,用于低分辨率环境下微表情的识别。

最后,我们总结了工作的贡献,并基于当前工作的局限性提出了关于微表情研究的未来计划。还计划将微表情和心率(可能还有来自面部的其他微妙信号)结合起来构建用于情感状态分析的多模式系统。

\keywords{微表情识别,监控视频,低分辨率,超分辨率,Fast LBP-TOP}% 中文关键词
%-
%-> 英文摘要
%-

\chapter{ABSTRACT}\chaptermark{ABSTRACT}% 摘要标题

\noindent

The face plays an important role in our social interactions as it conveys rich sources of information. We can read a lot from one face image, but there is also information we cannot perceive without special devices. The thesis concerns using computer vision methodologies to analyse subtle facial information that can hardly be perceived by naked eyes: the micro-expression.

Micro-expressions are rapid, involuntary facial expressions that belong to an essential non-verbal behavior which reveals emotions people do not intend to show. It is difficult for people to perceive Micro-expressions as they are too fast and subtle, thus automatic Micro-expressions analysis is a valuable work which may lead to important applications. For example, it has a wide range of applications in the national security and computer aided diagnosis, etc., which encourages us to conduct the research of automatic micro-expression recognition. However, the images captured from surveillance video easily suffer from the low-quality problem, which causes the difficulty in real applications. Due to the low quality of captured images, the existing algorithms are not able to perform as well as expected. For addressing this problem, we conduct a comprehensive study about the micro-expression recognition problem under low-resolution cases with face hallucination method. In the thesis, the progresses of Micro-expressions studies are reviewed, and four parts of work are described. (1) We introduced the existing Micro-expressions database, focusing on the first spontaneous Micro-expressions database, the SMIC. The lacking of data is hindering Micro-expressions analysis research. Since it is difficult to collect spontaneous Micro-expressions, the protocol for inducing and annotating SMIC is introduced to to help future Micro-expressions collections. (2) Introduced the Micro-expressions recognition framework based on traditional machine learning algorithm under low-resolution cases. Experiments were carried out on two mature databases, the SMIC and the CASME. The experimental results show that the proposed framework obtains promising results under low-resolution cases, which ends the situation where the existing algorithm cannot recognize the low-resolution Micro-expressions. (3) Extend the existing databases by data augmentation and dataset fusion, and introduce the most advanced convolutional neural network, the P3D ResNet, for image sequences to identify low-resolution Micro-expressions from the deep feature level. Compared with the algorithm based the traditional machine learning, the accuracy of this method is further improved. (4) A Micro-expression recognition system under low-resolution cases (LMERS) was proposed for the recognition of Micro-expressions under low-resolution cases.

At last, we summarize the contributions of the work, and propose future plans about Micro-expressions studies based on limitations of the current work. It is also planned to combine the Micro-expressions and heart rate (maybe also other subtle signals from face) to build a multimodal system for affective status analysis.

\englishkeywords{Micro-expression recognition, Surveillance video, Low-resolution, Super-resolution, Fast LBP-TOP}% 英文关键词
%---------------------------------------------------------------------------%
