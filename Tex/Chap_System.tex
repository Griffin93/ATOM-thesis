\chapter{系统设计}\label{chap:system}

% 为方便使用及更好地展示\LaTeX{}排版的优秀特性,ucasthesis的框架和文件体系进行了细致地处理,尽可能地对各个功能和板块进行了模块化和封装,对于初学者来说,众多的文件目录也许一开始让人觉得有些无所适从,但阅读完下面的使用说明后,会发现原来使用思路是简单而清晰的,而且,当对\LaTeX{}有一定的认识和了解后,会发现其相对Word类排版系统极具吸引力的优秀特性。所以,如果是初学者,请不要退缩,请稍加尝试和坚持,以领略到\LaTeX{}的非凡魅力,并可以通过阅读相关资料如\LaTeX{} Wikibook\citep{wikibook2014latex}来完善自己的使用知识。

\section{需求分析}

% \begin{enumerate}
%     \item 安装软件:根据所用操作系统和章节~\ref{sec:system}中的信息安装\LaTeX{}编译环境。
%     \item 获取模板:下载 \href{https://github.com/mohuangrui/ucasthesis}{ucasthesis} 模板并解压。ucasthesis模板不仅提供了相应的类文件,同时也提供了包括参考文献等在内的完成学位论文的一切要素,所以,下载时,推荐下载整个ucasthesis文件夹,而不是单独的文档类。
%     \item 编译模板:
%         \begin{enumerate}
%             \item Windows:双击运行artratex.bat脚本。
%             \item Linux或MacOS: {\scriptsize \verb|terminal| -> \verb|chmod +x ./artratex.sh| -> \verb|./artratex.sh xa|}
%             \item 任意系统:都可使用\LaTeX{}编辑器打开Thesis.tex文件并选择xelatex编译引擎进行编译。
%         \end{enumerate}
%     \item 错误处理:若编译中遇到了问题,请先查看“常见问题”(章节~\ref{sec:qa})。
% \end{enumerate}
%
% 编译完成即可获得本PDF说明文档。而这也完成了学习使用ucasthesis撰写论文的一半进程。什么?这就学成一半了,这么简单???,是的,就这么简单!

\section{功能设计}

\subsection{功能图}

% Thesis.tex为主文档,其设计和规划了论文的整体框架,通过对其的阅读可以了解整个论文框架的搭建。

\subsection{时序图}

% \begin{itemize}
%     \item Windows:双击Dos脚本artratex.bat可得全编译后的PDF文档,其存在是为了帮助不了解\LaTeX{}编译过程的初学者跨过编译这第一道坎,请勿通过邮件传播和接收此脚本,以防范Dos脚本的潜在风险。
%     \item Linux或MacOS:在terminal中运行
%         \begin{itemize}
%             \item \verb|./artratex.sh xa|:获得全编译后的PDF文档
%             \item \verb|./artratex.sh x|:快速编译模式
%         \end{itemize}
%     \item 全编译指运行 \verb|xelatex+bibtex+xelatex+xelatex| 以正确生成所有的引用链接,如目录,参考文献及引用等。在写作过程中若无添加新的引用,则可用快速编译,即只运行一遍\LaTeX{}编译引擎以减少编译时间。
% \end{itemize}

\subsection{等}

% 运行编译脚本后,编译所生成的文档皆存于Tmp文件夹内,包括编译得到的PDF文档,其存在是为了保持工作空间的整洁,因为好的心情是很重要的。

\section{界面设计}

\section{小节}
