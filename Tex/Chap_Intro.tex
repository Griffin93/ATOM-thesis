\chapter{绪论}\label{chap:introduction}

微表情(Micro Expressions),心理学名词,心理应激微反应的一部分,是人类表达自身情感信息的重要非语言性行为。微表情从人类本能出发,在大多数情况下,不受思想的控制,无法掩饰,也不能伪装\citep{Haggard1966}。因为它无法伪装的特性起初被人们用来作为鉴谎的辅助工具,随着人们对其不断深入的研究发现它在临床诊断、司法系统等有着很高的应用价值。近几年来随着计算机技术的不断发展,人们利用计算机视觉对微表情识别研究有了突飞猛进的成果,但就目前而言,还没有团队在低分辨率环境下对微表情做任何研究。本文从实际应用的角度出发,分析实际场景中面临的各种低质量问题,分别使用传统机器学习方法和深度学习方法对微表情识别。

本章主要阐述微表情识别研究的意义和低分辨率环境下微表情识别的重要性,国内外对微表情识别相关的研究和发展趋势,最后概述了文章的内容和结构分配。

\section{研究背景与意义}

达尔文在1872年出版了《The Expression of Emotions in Man and Animals》,从此拉开了人类对面部表情的系统性研究。时至今日,人类对面部表情的研究已经非常丰富与成熟,但主要关注的是显而易见的宏观表情(Macro Expressions),虽然在1966年Haggard和Isaacs首次提出了微表情现象(Micro-momentary Facial Expressions),但当时并未引起人们的普遍重视。直到三年后(1969年),Ekman和Friesen在临床发现了微表情,这一发现奠定了微表情在临床辅助治疗上的重要地位\citep{ekman1969nonverbal},也开启了微表情的研究热潮。

本节将从微表情研究的意义和计算机视觉对微表情研究的意义两方面分析。

\subsection{微表情研究的意义}

加利福尼亚大学洛杉矶分校的心理学教授Albert Mehrabian在上世纪六十年代发现了人际交流中的“55384”原则,他提出有效的沟通技巧应该包含三大要素:身体语言、声音和谈话内容\citep{Mehrabian1967Inference}。其中谈话内容传递的信息量是总信息量的7\%,声音(包括交谈时的语气、音调和音量)传递的信息量占总信息量的38\%,剩下的55\%来自身体语言(包括谈话期间身体姿势、肢体动作、面部表情、眼神和目光等),也就是说身体语言比谈话内容能传达更多有价值的信息。论文中阐明,出现这种情况的主要原因是谈话内容(口头语言)可以有意识地被控制,而身体语言这种非语言行为是无意识的举动,人类的主观意识很难控制动作语言行为。身体语言由三部分构成:表情语言、动作语言和空间语言。表情语言指的是通过面部肌肉运动和眼睛神态所传递出来的思想感情,动作语言指人类通过身体各个部位的动作或姿态来传递感情,空间语言主要指由个体与个体之间所保持的间距所形成的一种信息表达方式。在这三种身体语言中最容易被观察到的就是表情语言,艾伯特教授的这项发现说明了表情语言的重要性。

神经学家Paul Donald MacLean于上世纪五十年代提出了“大脑三位一体”理论(The Triune Brain),他认为人类颅腔内的脑并非只有一个,而是三个,这三个脑作为人类不同进化阶段的产物,按照出现顺序依次覆盖在已有的脑层之上,如同考古遗址一样\citep{Brain1999Kazlev}。根据在进化史上出现的先后顺序,他将人脑分成“爬行动物脑”(Reptilian brain)、“古哺乳动物脑”(Paleomammalian Brain)和“新哺乳动物脑”(Neomammalian Brain)三大部分,它们分别对应人脑的脑干(Archipallium)、边缘系统(Limbic System)和新皮质(Neocortex),它们共同控制着人类的身体行为。新皮质被称作“爱说谎的大脑”,经常会因为当事人的某种需要而出现说谎的现象。语言等由新皮质大脑控制的行为是不可信的,欺骗的嫌疑很大,想要得知对方内心的真实感受,必须观察对方边缘系统所控制的表情或肢体动作。边缘系统是控制人类情感的中心,管理着人类的非语言行为表达,因此是分析身体语言的重点。让人不加思索的产生本能反应是它的一大特点,它反映出了一个人最真实的一面,这很难被控制和掩饰。比如,当听到刺耳的噪音时你会不自主地捂住耳朵、手碰到高温或极寒物体时会马上缩回等。所以边缘系统的行为是诚实可信的行为,是人类的思想、感觉和意图的真实反应,也是人类生存、本能的反应,它属于微反应中除微语言以外的非语言行为反应,它包括了微动作(Micro Action)、微表情。

从上述例子可以看出,有关微表情的研究在心理学和神经学两大学科都有着充分的理论依据。Ekman等人在临床上发现微表情是来自于观看一位有自杀倾向的精神病患者的视频,视频中患者在回答医生问题时表现的很开心,没有任何想要自杀的异常迹象,但在随后的二次会谈中患者向医生承认其状况并未好转,而且她曾隐藏了自杀的计划。Ekman和 Friesen在逐帧慢放视频时发现确实存在两帧和绝望有关的负面表情,这与患者的二次会谈内容相吻合,但只持续了1/12 s。之后的几十年里Ekman和他的同事继续研究微表情,在不断的实践中量化并定义了微表情,这也引起了越来越多学术界和商业界人士的兴趣,目前微表情已经被应用到了众多领域,比如国家安全、司法系统、政治选举、临床诊断、公共管理和教育领域等\citep{Chiu2014The}。

\subsection{计算机视觉对微表情研究的意义}

我们人类是优秀的“人脸识别专家”,我们已经习惯甚至并没有意识到这一点。与其他类型的物种相比,我们人类为应对复杂的社交交互问题,大脑已经开发了特殊的识别脸部信息的功能模块,以便我们更好地从人脸中获取更丰富的信息,所谓“察言观色”就是很好的佐证。当然,人脸也是丰富的视觉信息的来源之处,我们可以从人脸中读取很多信息。比如眼前之人如果是著名人士,我们可以立即认出他或她,如果是陌生人,我们可以对这个人的性别、年龄、种族等做出基本正确的猜测,同时如果该人脸存在表情,我们也可以大致感知他或她的情绪状态。然而尽管我们是人脸识别专家,但这并不意味着我们已经解析出了全部的人脸信息,因为仍然存在部分无法用肉眼读取的深层次信息。

与其他感官(例如听觉和嗅觉)相比,我们的视觉认知功能在我们的大脑中更加精巧地被构建。然而,我们获取视觉信息的能力仍然受到生理机制的限制。超出我们感知范围的视觉变化(在空间域中太微妙或在时域中太快)将被我们的眼睛忽略。比如我们很难从人脸上观察到某个微表情,因为微表情会短暂且快速地发生,所涉及的肌肉运动强度也非常微弱,甚至表情发出者和观察者都察觉不到,尤其在高风险条件下微表情出现的机率更高,被察觉的可能性也更低。研究人员经过严密的统计,发现微表情持续时间最长为1/2秒而最短只有1/25秒,所涉及的肌肉运动强度更是微乎其微,而正常的表情(宏观表情)一般持续时间在1/2秒到5秒之间,有一个起承转合的过程。\citep{Yan2013How, matsumoto2011evidence, Porter2008Reading}。

由于微表情识别务实的使用价值,其提出者Ekman从2005年开始对英国情报机构、美国中央情报局等各国机构进行微表情识别培训,而那时他已经71岁高龄了\citep{Lie2010to}。他教辩护律师、健康专家、扑克选手,甚至对配偶心怀猜疑的人识破谎言,并且制作了网络课程。但他坦言人类对于微表情的识别能力终归是有限的,不仅要花费大量的人力和物力培训微表情识别专家,而且准确度不高,同时还伴有影响正常生活的风险\citep{Ekman2010A}。Ekman曾说自己的识谎能力影响到了日常生活,他从不试图去识破周围朋友、亲戚的微表情,“去揭露每个人的微表情,揭穿每个人的谎言,这只会让自己的生活痛苦万分”。

计算机的发明是为了帮助人类更好地处理人类不想去处理的任务,而识别人脸微表情这种会影响到日常生活的任务就是计算机存在的价值所在。而且当通过摄像机和计算机系统对待分析者分析时,不仅采集到的表情真实可靠(采集中采集对象并不知情,不存在任何干扰)而且通过计算机算法可以发现细微的人类无法察觉的表情变化,并且已经有大量的实验证明计算机的识别能力确实高于人类\citep{Li2017Towards}。我们可以更好更快的训练计算机完成人类能够完成的任务,如人脸检测、人脸识别,同时我们还可以训练计算机执行我们无法完成的任务,如捕获肉眼难以察觉的细微信息\citep{Doctoral2017Li}。

\section{国内外研究现状}\label{sec:system}

早期研究中,研究人员注重于测量或训练个体的微表情识别能力。Ekman 和Friesen在1974年制定了第一个微表情识别标准测验机制——BART(Brief Affect Recognition Test),但当时的微表情识别标准测验有着很大的缺陷,它所呈现的微表情是孤立的呈现,这与现实生活中微表情的动态呈现方式完全不相符,这样的测验没有任何生态效度\citep{Ekman1974Detecting, 殷明2016微表情}。1978年,Ekman发布了面部动作编码系统FACS(Facial Action Coding System),他们将人脸部的肌肉划分为43块,将它们随机组合获得了1万多种表情,但其中只有3000种具有情感意义,Ekman等人又根据人脸解剖学特点,将这43块肌肉划分成相互独立又相互联系的运动单元(Action Unit,AU),分析这些运动单元的运动特征和其所控制的主要区域,将这些信息与相关的表情匹配就能得出面部表情的标准运动\citep{ekman1978facial}。为了克服BART的缺陷, Matsumoto等人在2000年开发了更完善的微表情识别测量工具(Japanese and Caucasian Brief Affect Recognition Test,JACBART),该测验具有很好的可信度和严密的实验过程\citep{Matsumoto2000A}。Ekman等人在2002年根据日本人与高加索人短暂表情识别测验开发出了一个新的微表情识别训练工具METT(Micro Expression Training Tool),该训练工具有7种基本情绪的微表情,包括悲伤、恐惧、愤怒、厌恶、轻蔑、惊讶和高兴,METT 被应用在多种人群和领域,且对微表情受训者的识别能力有明显的提升\citep{ekman2003mett}。

除上述通过训练提升人工识别能力外自动地微表情识别系统也在如火如荼的发展中,研究者们已经开发出很多相关的算法,甚至在某些数据集的准确度可达90\%以上,但这些数据集有个明显的缺点,所有的微表情均为摆拍(Posed),这一缺点有其产生的必然性,但也严重违背了微表情的定义。为了解决这一问题,国内外的研究团队相继发表了自发微表情数据集,如中科院心理所分别在2013年和2014年发布了CASME\citep{Yan2013CASME}和CAMSE II\citep{Yan2014CASME}两个版本的数据集,后者比前者有着更高的时空分辨率和更多的数据量,但参与者全部为蒙古利亚人种(中国人),在数据的多样性上有一定的不足;芬兰奥卢大学的CMVS团队在2013年发布了SMIC数据集\citep{Li2013A},包括8名高加索人种和8名蒙古利亚人种,同时数据集中包含了高速视频数据(High speed video,HS)、近红外视频数据(Near infrared videos,NIR)和普通彩色视频数据(Normal color video,VIS);英国曼彻斯特城市大学在2017年发布了SAMM数据集\citep{Davison2018SAMM},是目前发表最新的数据集,包括了几乎全部的人种(蒙古利亚人种、高加索人种、尼格罗人种和大洋洲人种)和均衡的性别比,但其数据集只包含了高速灰度视频数据(见表~\ref{tab1})。优秀的数据集提供了良好的实验基础,自动微表情识别系统的研究主要集中在微表情检测(Micro-expression Spotting)和微表情识别(Micro-expression Recognition)。

\begin{table}[!htbp]
   \caption{自发微表情数据集比较}
   \label{tab1}
   \centering
   \footnotesize% fontsize
   \setlength{\tabcolsep}{4pt}% column separation
   \renewcommand{\arraystretch}{1.2}%row space
   \begin{tabular}{c|ccccccc}
     \toprule[0.7pt]
        & SMIC-HS &  SMIC-subHS &  SMIC-NIR &  SMIC-VIS &  CASME II &  SAMM \\ \hline
微表情片段    & 164       & 71         & 71       & 71       & 247           & 159             \\
参与者      & 16        & 8          & 8        & 8        & 26            & 32              \\
分辨率      & $640\times480$   & $640\times480$    & $640\times480$  & $640\times480$  & $640\times480$       & $2040\times1088$      \\
人脸分辨率    & $190\times230$   & $190\times230$    & $190\times230$  & $190\times230$  & $280\times340$       & $400\times400$         \\
FPS      & 100       & 100        & 100      & 100      & 200           & 200             \\
性别比(F/M) & 6/10      &  2/6          &  2/6        &  2/6        & 15/11         & 16/16           \\
FACS     & NO        & NO         & NO       & NO       & YES           & YES             \\
表情类      & 3         & 3          & 3        & 3        & 5             & 7               \\
平均年龄(SD) & 26.7(N/A) &   26.7(N/A) &  26.7(N/A) & 26.7(N/A)  & 22.03(SD=1.6) & 33.24(SD=11.32) \\
人种       & 2         & 2           &  2        & 2         & 1             & 4               \\
\bottomrule[0.7pt]
   \end{tabular}
\end{table}

微表情的检测指在一个图像序列(视频帧)中自动的检测微表情发生的起始时间间隔。在论文\citepns{Polikovsky2010Facial}中,作者以3D梯度描述为依据,提出了一种基于3D梯度投影描述检测微表情关键帧的方法,对微表情的发现提供了开创性的贡献。Wu等人提出使用Gabor滤波器构建一个自动的微表情识别系统,他们在METT训练集上达到了很高的检测性能\citep{Qi2011The}。Shreve等人使用基于应变的光流方法检测宏观表情和微表情\citep{Shreve2009Towards, Shreve2011Macro}。Li等人提出将特征差异(Feature Difference,FD)比较和峰值检测(Peak Detection,PD)相结合检测微表情,这种方法是第一个用于真实微表情数据集检测微表情且行之有效的方法。

微表情的识别是指在一段已经被确认包含微表情的图像序列里区分该段微表情属于具体的哪一类(如快乐、悲伤、厌倦等)。Polikovsky等人使用3D梯度描述符来识别AU标记的微表情,通过机器学习算法学习后分类\citep{Polikovsky2013Facial}。Wu等人结合Gentleboost和支持向量机(Support Vector Machine,SVM)分类器来识别来自METT的合成微表情样本。Pfister等人将时间插值模型(Temporal interpolation model,TIM)和多核学习(Multiple Kernel Learning,MKL)结合捕获图像序列的主要变化,利用三正交平面的局部二值模型(Local Binary Patterns on Three Orthogonal Planes,LBP-TOP)特征提取符提取动态纹理特征\citep{zhao2007dynamic},再用随机森林(Random forest,RF)作为分类器分类,这是第一个基于真实的微表情数据集上提出的微表情识别方法,该方法在SMIC数据集的第一版上实现的识别准确度达71.4\%(两类分类)\citep{pfister2011recognising}。Ruiz-Hernandez和Pietikäinen使用二阶高斯射流的重新参数化来生成更鲁棒的直方图,并且在SMIC数据集第一版上获得了更好的微表情识别结果\citep{ruiz2013encoding}。Song等人通过从面部和身体微弱运动中学习的稀疏编码来识别情绪,他们的微表情定义更加广泛,将身体部位(脸部除外)的姿势包括在内\citep{Song2013Learning}。Wang等人从张量独立色彩空间(不是普通RGB,Tensor independent color space,TICS)中提取LBP-TOP特征来识别微表情\citep{wang2014micro2}。Wang等人的另一篇论文中将局部时空方向特征与鲁棒主成分分析(Principal Component Analysis,PCA)的稀疏部分相结合一起用于微表情识别,在CASMEII上达到了65.4\%的准确度\citep{wang2014micro}。Huang等人为了提高微表情的辨别力,提出一种新的基于Laplacian的特征选择方法,在已发表的数据集中得到了很好的识别效果\citep{xiaohua2017discriminative}。Liong等人从光学应变量值得到一个时间段内人脸的细微相对位移量,并对局部特征赋予不同的权重,形成新的特征\citep{liong2014subtle}。Li等人将低强度的微表情视频经过欧拉视频放大,在三个正交平面上利用不同的特征提取符提取特征对微表情进行识别。随后的一段时间里大多都是在LBP-TOP特征提取符上做更改得到的变体版本,例如Huang等人通过在时空域中的积分投影获得被试者(目标)的形状属性,将人脸的形状属性与时空域上的纹理信息结合组成新的特征,即时空局部二值模式(Spatial-temporal local binary pattern,STLBP)\citep{Huang2015Facial}。Wang等人提出了六交叉点的LBP-TOP微表情识别特征描述符(LBP-Six Intersection Points,SIP),该方法是为了减少LBP-TOP中的冗余信息\citep{wang2014lbp}。Wang等人在随后的一年里进一步的又提出了一种紧实的LBP-TOP描述符(Super-compact LBP-Three Mean Orthogonal Planes,MOP),MOP所描述的紧实鲁棒形式不仅保留了基本模式而且减少了影响编码特征判别的冗余\citep{wang2015efficient}。Hong等人为提高LBP-TOP在时空信息上的计算效率,引入了张量的概念,这加速从三维空间到二维空间的实现过程\citep{FLBPTOP_2016}。Ngo等人提出了一种用于微表情图像序列预处理的选择性转移机(Selective Transfer Machine,STM),用于解决数据库中不平衡和不同面部形态的问题\citep{le2014spontaneous}。Lu等人发现微表情的图像序列在时空域中基于Delaunay三角归一化,提出了基于Delaunay的时间编码模型(Delaunay-based temporal coding model,DTCM)\citep{lu2014delaunay}。Oh等人通过Riesz小波变换获得多尺度单原信号,提取其幅值、相位、方向特征组成新的特征描述符进行微表情识别\citep{oh2015monogenic}。Liu等人提出了主方向平均光流法探索微表达的鉴别特征(Main direction Mean opticflow,MDMO),在利用鲁棒光流模型的同时考虑局部统计运动和空间位置信息\citep{liu2016mainl}。Li等人基于LBP-TOP的思想在三个正交平面上扩展了梯度方向直方图(HOG)和图像梯度方向直方图(HIGO)提出了HOG-TOP和HIGO-TOP。Xu等人利用光流估计对微表情图像序列选择的粒度进行像素级对齐,得到主光流方向,将其作为精细的面部动态特征描述符\citep{xu2017microexpression}。He等人提出了一种多任务的中层特征学习方法进行特征提取,该方法能够获得更具识别能力和泛化能力的中层特征\citep{he2017multi}。近年来,一些研究者利用时下流行的深度学习算法从微表情中学习深度特征,但效果并不理想。主要原因是深度学习算法需要大量的训练样本,但当前数据集的规模非常有限\citep{patel2016selective}。

\section{本文的研究内容}

本文简单介绍了微表情识别的研究现状和基本方法,以及其不容小觑的应用价值。然而,目前的微表情研究都是基于高质量数据集的基础上展开的,例如SMIC数据集的人脸分辨率为$190\times230$像素,最新发布的SAMM数据集的人脸分辨率达到$400\times400$像素之高,但在实际应用中由于图像采集设备的机能限制,难免会遇到低像素的视频数据,这严重影响了几乎所有的微表情识别算法的性能,为了解决这一问题,本文提出了专门针对低分辨率环境下的微表情识别方法:按照常规方法对视频做预处理,然后再使用超分辨重建技术从低维图像中近似的重建出高维图像,对重建出的高维图像进行微表情识别,最后比较重建后的微表情识别效率与未重建前的低维图像的识别效率,同时分别从传统机器学习和深度学习两个角度介绍了系统框架和识别结果。

本文共分为六章,内容具体安排如下:

第一章:绪论。阐述了微表情研究的背景及意义,概括了国内外最新的微表情研究现状,最后对文章的整体结构做出安排;

第二章:相关工作。简单介绍了微表情的概念和微表情数据集,总结了微表情识别在传统方法和深度学习方法的进展,阐述了低分辨率微表情的识别的意义和最新进展;

第三章:基于传统方法的低分辨率环境下微表情识别的研究。主要介绍了人脸对齐与分割,其中包括对单一目标检测中主动形状模型(Active Shape Model,ASM)准确度的算法改进,图像序列的超分辨重建,LBP-TOP和SVM;

第四章:基于深度学习的低分辨率环境下微表情识别的研究。主要介绍了伪3D残差网络(Pseudo-3D Residual Networks,P3D ResNet),数据增强以及实验分析;

第五章:低分辨率环境下微表情识别可视系统。通过对需求分析设计出系统的时序图和功能图,最后设计出功能完善的可视化系统;

第六章:总结与展望。总结全文的工作,分析不足和展望未来前景。
